\documentclass[12pt]{article}
\usepackage[top=1in, bottom=1in, left=1in, right=1in]{geometry}
%\usepackage[margin=1in]{geometry}
\usepackage[onehalfspacing]{setspace}
%\usepackage[doublespacing]{setspace}
\usepackage{amsmath, amssymb, amsthm}
\usepackage{enumerate, enumitem}
\usepackage{fancyhdr, graphicx, proof, comment, multicol}
\usepackage[none]{hyphenat} % This command prevents hyphenation of words

%    Good website with common symbols
% http://www.artofproblemsolving.com/wiki/index.php/LaTeX%3ASymbols
%    How to change enumeration using enumitem package
% http://tex.stackexchange.com/questions/129951/enumerate-tag-using-the-alphabet-instead-of-numbers
%    Quick post on headers
% http://timmurphy.org/2010/08/07/headers-and-footers-in-latex-using-fancyhdr/
%    Info on alignat
% http://tex.stackexchange.com/questions/229799/align-words-next-to-the-numbering
% http://tex.stackexchange.com/questions/43102/how-to-subtract-two-equations
%    Text align left-center-right
% http://tex.stackexchange.com/questions/55472/how-to-make-text-aligned-left-center-right-in-the-same-line
\usepackage{microtype} % Modifies spacing between letters and words
\usepackage{mathpazo} % Modifies font. Optional package.
\usepackage{mdframed} % Required for boxed problems.
\usepackage{parskip} % Left justifies new paragraphs.
\linespread{1.1} 


\newenvironment{problem}[1]
{\begin{mdframed}[linewidth=0.6pt]
        \textsc{Problem #1:}

}
    {\end{mdframed}}

\newenvironment{solution}
    {\textsc{Solution:}\\}
    {\newpage}% puts a new page after the solution
    
\newenvironment{statement}[1]
{\begin{mdframed}[linewidth=0.6pt]
        \textsc{ #1:}

}
    {\end{mdframed}}

%\newenvironment{prf}
 %   {\textsc{Proof:}\\}
 %   {\newpage}% puts a new page after the solution

\newcommand{\R}{\mathbb{R}}
\newcommand{\C}{\mathbb{C}}
\newcommand{\Z}{\mathbb{Z}}
\newcommand{\N}{\mathbb{N}}
\newcommand{\Q}{\mathbb{Q}}

\begin{document}

\noindent
\textbf{ECS132} \hfill \textbf{Atharva Chalke, Sam Yuan} \\
\normalsize Prof. Matloff \hfill Due Date: 10/14/19 \\


\begin{center}
\textbf{Homework 2}
\end{center}

% This is how you call the environment for the statement to be proved.
\begin{statement}{Problem A}
Think of the points 1, 2, ..., n on the number line. Each is either "occupied" or not, with probability p, independently, except that points 1 and n are occupied for sure.

Let M denote the minimum distance between successive occupied points. E.g., if n is 6 and points 1, 5 and 6 are occupied, then M = 1.

1. Find Var(M) for the case n = 5, in terms of p.\\
2. Write a function with call form
simA(nreps,n,p)
to find Var(M) for the general case by simulation.

The answer for n = 5 and p = 0.4 is 1.46.
\end{statement}

% This is how you call the proof environment
\subsection*{Solution}
\textbf{Part 1 }\\\\
\textbf{Definitions}\\
Let $A_i$ be 1 if the point i is occupied else 0.\\
Let Support of M be D, Thus D = $\{1,2,4\}$
\begin{align}
E(M) & = \sum_{d \in D} d * P(M = d) \nonumber 
\end{align}
\begin{align}
P(D = 4) & = P(A_2 = 0 and A_3 = 0 and A_4 = 0) \nonumber \\
 & = (1-p)^3 \nonumber 
\end{align}
\begin{align}
P(D = 2) & = P(A_2 = 0 and A_3 =1 and A_4 = 0) \nonumber \\
 & = p*(1-p)^2 \nonumber 
\end{align}
Every other combination of the 3 spaces in between results in a distance of 1. To find the probability we find the probability for each case and sum them up.\\\\
\begin{center}
    
    \begin{tabular}{|c|c|}
    \hline 
    Combination  &  Probability \\
    \hline
    001 & $p * (1-p)^2$ \\
    011 & $p^2*(1-p)$ \\
    100 & $p*(1-p)^2$ \\
    101 & $p^2*(1-p)$ \\
    110 & $p^2*(1-p)$ \\
    111 & $p^3$  \\
    \hline
    \end{tabular}
\end{center}
\begin{align}
P(D = 1) & =    (1-p)^2*p +  p^2*(1-p) + p*(1-p)^2 + p^2*(1-p) + p^2*(1-p) + p^3 \nonumber \\
& = 2*p*(1-p)^2 + 3 *p^2*(1-p) + p^3 \nonumber
\end{align}

\begin{align}
E(M) & = 4 * (1-p)^3 + 2 * p*(1-p)^2 + 1 * (2*p*(1-p)^2 + 3 *p^2*(1-p) + p^3) \nonumber
\end{align}
Lets find variance now.\\
\begin{align}
Var(M) & = \sum_{d \in D} (d - E(M))^2 * P(M = d) \nonumber \\
& = [(1 - E(M))^2 * (2*p*(1-p)^2 + 3 *p^2*(1-p) + p^3)] + [(2-E(M))^2 * p*(1-p)^2] \nonumber \\
& \qquad + [(4-E(M))^2 * (1-p)^3 ] \nonumber
\end{align}
Substituting the value for p, i.e. 0.4\\
Var(M) = 1.460736

\textbf{Part 2 }\\
Code2.R
\newpage
\begin{statement}{Problem B}
In the setting of Problem B, Homework I, let Wk be the number of reported 0 bits among the first k bits. Using indicator functions as in Equations (3.98) and following (this is required), find EWk in terms of k, p and q.
\end{statement}





`

\newpage
\begin{statement}{Problem D}
    This problem involves Chebychev's Inequality, Equation (3.63). (You are not responsible for this material on quizzes, but it will be used here.)\\

    Evaluate both sides of (3.63) for the following case: X has a binomial distribution with 10 trials and success probability 0.25, and c=2. Verify that the inequality does hold in this case.
\end{statement}

% This is how you call the proof environment
\subsection{Solution}
Given Chebychev's Inequality, p = 0.25, c = 2\\\\
For variance:
\begin{equation}
    \begin{aligned}
        V_{ar}(X)&=np(1-p)\\
        &=10*0.25*(1-0.25)\\
        &=1.875
    \end{aligned}
\end{equation}
For mean:
\begin{equation}
    \begin{aligned}
        \mu&=np\\
        &=10*0.25=2.5
    \end{aligned}
\end{equation}
For c = 2:
\begin{equation}
    \dfrac{1}{c^2}=\dfrac{1}{4}
\end{equation}
Therefore:
\begin{equation}
    \begin{aligned}
        P(|X-\mu|\geq cV_{ar})&=P(|X-2.5|\geq 2*1.875)\\
        &=P(|X-2.5|\geq 3.75)\\
        &=\sum\limits_{i=7}^{10}P(X=i)\\
        &=\sum\limits_{i=7}^{10}\binom{n}{k}p^k(1-p)^{n-k}\\
        &=\binom{10}{7}0.25^7(1-0.25)^{10-7}+\binom{10}{8}0.25^8(1-0.25)^{10-8}\\
        &\ \ \ \ \ \ +\binom{10}{9}0.25^9(1-0.25)^{10-9}+\binom{10}{10}0.25^{10}(1-0.25)^{10-10}\\
        &=0.0035\leq\dfrac{1}{4}
    \end{aligned}
\end{equation}
Hence, Chebychev's inequality does hold in this case.

\end{document}
%--------------%
