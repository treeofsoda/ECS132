\documentclass[12pt]{article}
\usepackage[top=1in, bottom=1in, left=1in, right=1in]{geometry}
%\usepackage[margin=1in]{geometry}
\usepackage[onehalfspacing]{setspace}
%\usepackage[doublespacing]{setspace}
\usepackage{amsmath, amssymb, amsthm}
\usepackage{enumerate, enumitem}
\usepackage{fancyhdr, graphicx, proof, comment, multicol}
\usepackage[none]{hyphenat} % This command prevents hyphenation of words

%    Good website with common symbols
% http://www.artofproblemsolving.com/wiki/index.php/LaTeX%3ASymbols
%    How to change enumeration using enumitem package
% http://tex.stackexchange.com/questions/129951/enumerate-tag-using-the-alphabet-instead-of-numbers
%    Quick post on headers
% http://timmurphy.org/2010/08/07/headers-and-footers-in-latex-using-fancyhdr/
%    Info on alignat
% http://tex.stackexchange.com/questions/229799/align-words-next-to-the-numbering
% http://tex.stackexchange.com/questions/43102/how-to-subtract-two-equations
%    Text align left-center-right
% http://tex.stackexchange.com/questions/55472/how-to-make-text-aligned-left-center-right-in-the-same-line
\usepackage{microtype} % Modifies spacing between letters and words
\usepackage{mathpazo} % Modifies font. Optional package.
\usepackage{mdframed} % Required for boxed problems.
\usepackage{parskip} % Left justifies new paragraphs.
\linespread{1.1} 


\newenvironment{problem}[1]
{\begin{mdframed}[linewidth=0.6pt]
        \textsc{Problem #1:}

}
    {\end{mdframed}}

\newenvironment{solution}
    {\textsc{Solution:}\\}
    {\newpage}% puts a new page after the solution
    
\newenvironment{statement}[1]
{\begin{mdframed}[linewidth=0.6pt]
        \textsc{ #1:}

}
    {\end{mdframed}}

%\newenvironment{prf}
 %   {\textsc{Proof:}\\}
 %   {\newpage}% puts a new page after the solution

\newcommand{\R}{\mathbb{R}}
\newcommand{\C}{\mathbb{C}}
\newcommand{\Z}{\mathbb{Z}}
\newcommand{\N}{\mathbb{N}}
\newcommand{\Q}{\mathbb{Q}}

\begin{document}

\noindent
\textbf{ECS132} \hfill \textbf{Atharva Chalke} \\
\normalsize Prof. Matloff \hfill Due Date: 10/14/19 \\


\begin{center}
\textbf{Homework 1}
\end{center}

% This is how you call the environment for the statement to be proved.
\begin{statement}{Problem A}
This will be a variant of the bus ridership problem, Sec. 2.11.

In addition to individual riders boarding the bus, there may be pairs, e.g. parent and child. At any stop, either 0 pairs or 1 pair will board, with probability 0.4 and 0.6, respectively. Similarly, any pair already on board will alight at a stop, with probability 0.2. Pairs act independently from individuals and from other pairs.

1. Find P(L2 = 0).

2. A newspaper photograph of the bus arriving at the second stop shows a passenger alighting. Use simulation to find the approximate probability that this passenger was part of a pair. Extra Credit: Do this problem mathematically.
\end{statement}

% This is how you call the proof environment
\subsection*{Solution}
\textbf{Part 1 }\\
\textbf{Definitions}\\
Let L$_i$ denote the number of passengers on the bus as it leaves the $i^{th}$ stop, i = 1,2,3......\\
Let $B_i$ denote the number of individual new passengers who board the bus at the $i^{th}$ stop.\\
Let $R_i$ denote the number of paired new passengers who board the bus at the $i^{th}$ stop.\\\\
\textbf{Assumptions}\\
The pairs who board together, also leave together.\\
Passengers who board at a stop do not get off at the same stop.\\
\begin{align}
P(L_2=0) & = P(B_1=0\ and\ R_1=0\ and\ L_2= 0)\ +  P(B_1=0\ and\ R_1=2\ and\ L_2= 0)\  \nonumber \\
  & \qquad + P(B_1=1\ and\ R_1=0\ and\ L_2= 0)\ + P(B_1=1\ and\ R_1=2\ and\ L_2= 0)\  \nonumber \\
  & \qquad + P(B_1=2\ and\ R_1=0\ and\ L_2= 0)\ + P(B_1=2\ and\ R_1=0\ and\ L_2= 0)\  \nonumber \\
  & = \sum_{i=0}^{2} P(B_1=i\ and\ R_1=0\ and\ L_2= 0)\ +  P(B_1=i\ and\ R_1=2\ and\ L_2= 0) \nonumber
\end{align}
Lets solve this for each i
\newpage
\textbf{i = 0 }\\
\begin{align}
 & = P(B_1=0\ and\ R_1=0\ and\ L_2= 0)\ +  P(B_1=0\ and\ R_1=2\ and\ L_2= 0)   \nonumber \\
 & = P(B_1=0\ and\ R_1=0\ ) * P(L_2=0 |B_1=0\ and\ R_1=0\ )    \nonumber \\
 & \qquad + P(B_1=0\ and\ R_1=2\ ) * P(L_2=0 |B_1=0\ and\ R_1=2\ )\quad [Using  P(A and B) = P(A) * P(B|A)] \nonumber \\
 & = 0.5 * 0.4*0.5*0.4\ + 0.5*0.6*0.5*0.4*0.2 \nonumber\\
 & = 0.052 \nonumber
\end{align}
\textbf{i = 1 }\\
\begin{align}
 & = P(B_1=1\ and\ R_1=0\ and\ L_2= 0)\ +  P(B_1=1\ and\ R_1=2\ and\ L_2= 0)   \nonumber \\
 & = P(B_1=1\ and\ R_1=0\ ) * P(L_2=0 |B_1=1\ and\ R_1=0\ )    \nonumber \\
 & \qquad + P(B_1=1\ and\ R_1=2\ ) * P(L_2=0 |B_1=1\ and\ R_1=2\ )\quad [Using  P(A and B) = P(A) * P(B|A)] \nonumber \\
 & = 0.4 * 0.4 * 0.5*0.4*0.2\ + 0.4*0.6*0.5*0.4*0.2*0.2 \nonumber\\
 & = 0.00832 \nonumber
\end{align}
\textbf{i = 2 }\\
\begin{align}
 & = P(B_1=2\ and\ R_1=0\ and\ L_2= 0)\ +  P(B_1=2\ and\ R_1=2\ and\ L_2= 0)   \nonumber \\
 & = P(B_1=2\ and\ R_1=0\ ) * P(L_2=0 |B_1=2\ and\ R_1=0\ )    \nonumber \\
 & \qquad + P(B_1=2\ and\ R_1=2\ ) * P(L_2=0 |B_1=2\ and\ R_1=2\ )\quad [Using  P(A and B) = P(A) * P(B|A)] \nonumber \\
 & = 0.1*0.4*0.4*0.5*0.2^2\ + 0.6*0.1*0.5*0.4*0.2^3  \nonumber\\
 & = 0.000416 \nonumber
\end{align}
\textbf{Final Answer}\\
Summing up these values we get 
\begin{align}
    & = 0.052 + 0.00832 + 0.000416 \nonumber\\
    & = 0.060736 \nonumber
\end{align}
\textbf{Part 2}\\
\textbf{Assumptions}\\
1. If both, the individual passenger and the pair gets down, the pair gets down first.\\
2. The photograph was taken of the first person getting down.\\\\
\textbf{Definitions}\\
Let X$_i$ be the number of pairs getting down at stop i.\\
Let B$_i$ be the number of pairs getting on the bus at stop i.
\begin{align}
P(X_2=1) & = P(B_1=0\ and\ X_2=1)\ +  P(B_1=1\ and\ X_2=1)\  \nonumber \\
 & = 0.4*0 + 0.6*0.2 \nonumber \\
 & = 0.12 \nonumber
\end{align}

\newpage
%------------%
\begin{statement}{Problem B}
Consider a communications line in which the true bits are 1 or 0, independently with probability of a 1 being p each. At any given bit, the line will fail, independently with probability q. Once the line has failed, it stays failed until it is repaired, reporting each bit as 0 regardless of the bit's true value. Of course, a long string of 0s should make us suspicious and cause us to inspect the line. Let Bi denote the actual value of the ith bit , and Ri the reported value, i = 1,2,3,...

Your answers to Parts 1 and 2 must be in closed form, i.e. no Σ and the like; you may need to use Properties of Geometric Series, pp.73-74. Part 4 also has a closed-form answer.

1. Find P(Bi = Ri) for i = 1,2,3,...\\

2. Say you have software monitoring the line, which will flag a possible problem whenever it observes k consecutive 0s after a 1 (i.e. a 1 followed by k 0s, the last of which is the most recent bit). Find the probability that a flag is raised at bit r, r = k+1,k+2,k+3,... This is a single expression in p, q , k and r.\\

3. Write a function with call form
simline(nreps,p,q,k,r)
that finds via simulation the probability that a flag is raised at bit r but not before that time.\\

4. The $B_i$ are independent, but intuitively, the $R_i$ are not. Show that to be the case by calculating P($R_1$ = 0 and $R_2$ = 0), P($R_1$ = 0) and P($R_2$ = 0), and noting that the product of the latter two probabilities is not equal to the first one.\\
Your answer must be consist of general expressions in p and q. With those equal to 0.6 and 0.2, respectively, one gets the answers 0.36640 and 0.32032.

\end{statement}
\subsection*{Solution}
\textbf{Part 3}\\
Code: Code1.R\\
NOTE: numbering starts from 1, i.e., the first bit is bit 1 and not bit 0.\\\\
\textbf{Part 4 }\\
\textbf{Subsection 1: $P(R_1 = 0\ and\ R_2 = 0)$ }\\
\textbf{Definitions}\\
Let $Z_i$ be false if the line failed for bit i and true if it didn't fail.\\
Let $B_i$ denote the actual value of the ith bit \\
Let  $R_i$ the reported value, i = 1,2,3,...
\begin{align}
P(R_1 = 0\ and\ R_2 = 0)  & = P(R_2 = 0\ and\ Z_1 = False\ or R_2 = 0\ and\ Z_1 = True \nonumber \\
 & \qquad and  B_1 = 0\ and\ R_1 = 0)\ \nonumber \\
 & = P(R_2 = 0\ and\ Z_1 = False) + P(R_2 = 0\ and\ Z_1 = True\ \nonumber \\
 & \qquad and\  B_1 = 0\ and\ R_1 = 0)\ \nonumber \\
 & \qquad [using\ P(A or B) = P(A) + P(B)]\ \nonumber
\end{align}

Solving $ P(R_2 = 0\ and\ Z_1 = False)$
\begin{align}
P(R_2 = 0\ and\ Z_1 = False)  & = P(Z_1 = False) * P(R_2 = 0 | Z = False) \nonumber \\
& [using\ P(A and B) = P(A) * P(B|A)]\ \nonumber \\
& = q * 1 \nonumber
\end{align}

Solving $P(R_2 = 0\ and\ Z_1 = True\ and\  B_1 = 0\ and\ R_1 = 0) $ \\
\begin{align}
P(R_2 = 0\ and\ Z_1 = True\ and\  B_1 = 0\ and\ R_1 = 0)  & = P(Z_1 = True\ and\  B_1 = 0 and\ R_1 = 0) * \nonumber \\
& \qquad P( R_2 = 0| Z_1 = True\ and\  B_1 = 0\ and\ R_1 = 0) \nonumber \\
& \qquad[using\ P(A and B) = P(A) * P(B|A)] \nonumber
\end{align}
Lets solve the first sub-part of this equation\\
\begin{align}
P(Z_1 = True\ and\  B_1 = 0 and\ R_1 = 0) & = P(B_1 = 0) * P(R_1 = 0\ and\ Z_1 = True | B_1 = 0 ) \nonumber \\
& \qquad [Using\ P(A and B) = P(A) * P(B|A) ] \nonumber \\
& = (1-p) * (1-q) \nonumber
\end{align}
Lets solve the second sub-part of this equation\\
$P( R_2 = 0| Z_1 = True\ and\  B_1 = 0\ and\ R_1 = 0)$ is the same as $P(R_1 = 0)$ that is solved for in the next part for $P(R_1 = 0)$. It comes out to be $(1-p)*1 + p*q$.\\
\textbf{Final answer}\\
$P(R_1 and R_2) = q*1+  ((1-p)*(1-q)*(1-p+p*q)) $
For p = 0.6 and q = 0.2 we get 0.3664.\\\\
\textbf{Subsection 2: $P(R_1 = 0) * P( R_2 = 0)$ }\\
\textbf{Definitions}\\
Let $Z_i$ be false if the line failed for bit i and true if it didn't fail.\\
Let $B_i$ denote the actual value of the ith bit \\
Let  $R_i$ the reported value, i = 1,2,3,...\\\\
Lets solve each one of the parts of this equation.
\begin{align}
P(R_1 = 0) & = P(B_1 = 0\ and R_1 = 0\ or\ B_1 = 1\ and\ R_1 = 0)\ \nonumber\\
& = P(B_1 = 0\ and R_1 = 0) + P(B_1 = 1\ and R_1 = 0) \nonumber \\
& \qquad [Using P(A or B) = P(A) + P(B)] \nonumber \\
& = P(B_1 = 0)*P(R_1 = 0 | B_1 = 0) + P(B_1) * P(R_1 = 0 |B_1 = 1 ) \nonumber \\
& \qquad [Using P(A and B) = P(A) * P(B|A)] \nonumber \\
& = (1-p)*1 + p*q \nonumber
\end{align}
\begin{align}
P(R_2 = 0) & = P(R_2 = 0\ and\ B_2 = 0\ and\ Z_1 = False\ or\ R_2 = 0\ and\ B_2 = 0\  and\ \nonumber \\
& \qquad Z_1 = True\ or\ R_2 = 0\ and\ B_2 = 1\ and\ Z_1 = False\ or R_2 = 0\ and\ \nonumber \\ 
& \qquad B_2 = 1\ and\ Z_1 = True\ )  \nonumber \\
& = P(R_2 = 0\ and\ B_2 = 0\ and\ Z_1 = False) +  P(R_2 = 0\ and\ B_2 = 0\ \nonumber \\
& \qquad and\ Z_1 = True) +  P(R_2 = 0\ and\ B_2 = 1\ and\ Z_1 = False) +  \nonumber \\
& \qquad P(R_2 = 0\ and\ B_2 = 1\ and\ Z_1 = True) \nonumber \\
& \qquad [Using P(A or B) = P(A) + P(B)] \nonumber
\end{align}
Lets solve each of these. We use P(A and B) = P(A) * P(B|A) for each one of them.\\
\begin{align}
P(R_2 = 0\ and\ B_2 = 0\ and\ Z_1 = False) & = P(B_2 = 0\ and\ Z_1 = False) * P(R_2 = 0 | B_2 = 0 \nonumber \\
&  \qquad and Z_1 = False \nonumber) \\
& = (1-p) * (1-q) \nonumber
\end{align}
\begin{align}
P(R_2 = 0\ and\ B_2 = 0\ and\ Z_1 = True) & = P(B_2 = 0\ and\ Z_1 = True) * P(R_2 = 0 | B_2 = 0 \nonumber \\
&  \qquad and Z_1 = True \nonumber) \\
& = (1-p)*q*1 \nonumber
\end{align}
\begin{align}
P(R_2 = 0\ and\ B_2 = 1\ and\ Z_1 = False) & = P(B_2 = 1\ and\ Z_1 = False) * P(R_2 = 0 | B_2 = 1 \nonumber \\
&  \qquad and Z_1 = False \nonumber) \\
& = p*(1-q)*q \nonumber
\end{align}
\begin{align}
P(R_2 = 0\ and\ B_2 = 1\ and\ Z_1 = True) & = P(B_2 = 1\ and\ Z_1 = True) * P(R_2 = 0 | B_2 = 1 \nonumber \\
&  \qquad and Z_1 = True \nonumber) \\
& = p*q*1 \nonumber
\end{align}
Therefore,
\begin{align}
P(R_2 = 0) = (1-p) * (1-q) + (1-p)*q*1 + p*(1-q)*q + p*q*1 \nonumber
\end{align}
Substituting p = 0.6 and q = 0.2, we get the following:\\
$P(R_1 = 0)$ = (1-p) + p*q = 0.52\\
$P(R_2 = 0)$ = (1-p) * (1-q) + (1-p)*q*1 + p*(1-q)*q + p*q*1 = 0.616\\
$P(R_1 = 0)*P(R_2 = 0)$ = 0.52 * 0.616 =  0.32032
\newpage
%------------%
\begin{statement}{Problem C}
This problem will be similar to the broken rod example, Sec. 2.14.10.

Say we have a square plate, length 1.0 on each side. The plate is dropped, and breaks into two pieces, as follows: A break point occurs at a random point in the square (call runif() twice), and then along a random angle between 0 and π.

Write a function with call form

simplate(nreps,p)
that finds by simulation the probability that the smaller piece has area less than p.
\end{statement}
\subsection*{Solution}
Code: Code1.R




`


\end{document}
%--------------%
